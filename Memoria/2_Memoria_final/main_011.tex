\section{Descripción general del proyecto} % (fold)
\label{sec:descripcion_general_del_proyecto}
    % Usuarios
	\subsection {Usuarios}
	En primer lugar se muestra una clasificación de los posibles usuarios del sistema. De esta forma podremos identificar y asociar cada característica con cada uno de ellos. 

	\begin{itemize}
	\item \textbf{Médicos.} A ellos está dedicada la aplicación. Pueden tener cualquier tipo de especialidad.
	\item \textbf{Pacientes.} No son los usuarios principales del sistema, pero son el sustento de los médicos. Pueden interactuar con el software a la hora de reservar citas o de realizar pagos por adelantado.
	\item \textbf{Administrador del sistema.} Encargado de verificar que el sistema funciona correctamente. Además, debe verificar que un médico esté licenciado como tal y realizar las copias de seguridad.
	\end{itemize}

	% Objetivos
	\subsection{Objetivos}
	Se propone diseñar e implementar una aplicación SaaS (Software as a Service), que lo que pretende es ofertar un {\bf software como servicio} a todos aquellos especialistas sanitarios interesados. 

		Por tanto, se desarrollará una aplicación web, en la que se abordaran los siguientes módulos:
	\begin{itemize}
		\item \textbf{Gestión de médicos.} Funcionalidades que pueden desarrollar los distintos especialistas para gestionar su consulta de la forma más eficiente posible. Tienen que ver principalmente con su calendario, sus plantillas, sus pacientes y con diversos datos de configuración.
		\item \textbf{Gestión de pacientes.} Funcionalidades que pueden desarrollar los usuarios con el rol de paciente. Tienen que ver principalmente con sus citas, sus médicos, con diversos datos de configuración y con la información de sus fichas médicas.
		\item \textbf{Gestión de fichas médicas.} Ficha médica de los pacientes, con sus pruebas médicas, diagnósticos, tratamientos, informes, información asociada, etcétera.
		\item \textbf{Panel de administrador.} El sistema será gestionado y mantenido por un administrador, el cuál podrá realizar diversas tareas.
		\item \textbf{Gestión de actividades generales.} Una serie de funcionalidades que podrán realizar usuarios sin necesidad de estar identificados, tales como darse de alta en la aplicación, ver las condiciones generales, un tour de la aplicación, buscar si un médico forma parte del sistema, etcétera.
	\end{itemize}
	 
% subsection descripción_general_del_proyecto (end)
