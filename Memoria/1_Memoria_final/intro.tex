% Capítulo de Introducción
\chapter*{Introducción} % (fold)
\label{cha:introduccion}
	El {\bf dominio} de la aplicación es el de las {\bf consultas médicas privadas}, donde normalmente existen {\bf dos actores} principales: el médico y el paciente. Es decir, el proyecto está orientado a médicos que gestionen consultas de carácter particular y a sus pacientes.

		En primer lugar debemos ponernos en situación. En el sector de la sanidad privada, la gestión de centros médicos (desde el seguimiento de un paciente hasta la organización interna del mismo) se ha desarrollado {\bf tradicionalmente en papel} (información analógica). Sin embargo, con el avance tecnológico y con la aparición de las nuevas tecnologías de información y comunicación (TICs), se crean nuevos modelos de gestión y administración de los centros, pasando todos los flujos de datos a versión digital. Es decir, la información comienza a gestionarse usando software específico, pero {\bf centralizado en una localización física concreta}. A pesar de todo, sigue habiendo muchos especialistas que no han dado el paso y continúan utilizando métodos arcaicos.

		Por otro lado, {\bf el proceso de concertar una cita} sigue siendo el mismo que antaño: un paciente debe presentarse en la consulta del médico físicamente y solicitar cita ó bien debe realizar una llamada telefónica para ser atendido por un administrativo.

		Por tanto, con el objetivo de poder gestionar digitalmente y desde “la nube” todos los aspectos referentes al flujo de información de una consulta médica, se requiere desarrollar una aplicación web desde la que se gestionen un gran número de consultas privadas, con todos los datos referentes a médicos (curriculum, precios, horarios de disponibilidad, etc.), a pacientes (ficha médica, diagnósticos, pruebas, exploraciones, etc.), los pagos y el seguimiento de la contabilidad y además, permita que sean los propios especialistas los que muestren a los pacientes de que horas libres disponen, para que sean estos últimos los encargados de asignarse, desde cualquier ubicación en la que exista un computador con internet, la cita que más les convenga.
% chapter introducción (end)